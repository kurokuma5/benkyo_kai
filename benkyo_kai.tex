\documentclass[dvipdfmx,draft,11pt]{jsarticle}
\usepackage{booktabs}
\usepackage[final]{graphics}
\begin{document}



\title{大阪高裁判決}
\author{kurokuma}
\date{2018年1月}
\maketitle



\begin{quotation}
【判例時報2345-93】\\

\end{quotation}

\section{条文}

\subsection{商標法2条3項8号}

3項「この法律で標章について『使用』とは、次に掲げる行為をいう。」\\

8号
\begin{quote}
「商品若しくは役務に関する広告、価格若しくは取引書類に標章を付して展示し、若しくは頒布し、又はこれらを内容とする情報に標章を付して電磁的方法により提供する行為」

\end{quote}

\subsection{不正競争防止法2条1項1号}


1項「この法律において『不正競争』とは、次に掲げるものをいう。」

1号
\begin{quote}
「他人の商品等表示(人の業務に係る氏名、商号、商標、標章、商品の容器若しくは包装その他の商品又は営業を表示するものをいう。以下同じ。)として需要者の間に広く認識されているものと同一若しくは類似の商品等表示を使用し、又はその商品等表示を使用した商品を譲渡し、引き渡し、譲渡若しくは引渡しのために展示し、輸出し、輸入し、若しくは電気通信回線を通じて提供して、他人の商品又は営業と混同を生じさせる行為」
\end{quote}

\section{当事者}

\[ 原告_(株)生活と科学社 \to 被告_楽天(株) \]


\subsection{原告_(株)生活と科学社}

\begin{itemize}
	\item 石けんの販売等を業とする株式会社
	\item 「石けん百貨」等の商標権を保有していた。
\end{itemize}






 \flushright{以 上}
 
\end{document}
